% Options for packages loaded elsewhere
\PassOptionsToPackage{unicode}{hyperref}
\PassOptionsToPackage{hyphens}{url}
\PassOptionsToPackage{dvipsnames,svgnames,x11names}{xcolor}
%
\documentclass[
  letterpaper,
  DIV=11,
  numbers=noendperiod]{scrartcl}

\usepackage{amsmath,amssymb}
\usepackage{iftex}
\ifPDFTeX
  \usepackage[T1]{fontenc}
  \usepackage[utf8]{inputenc}
  \usepackage{textcomp} % provide euro and other symbols
\else % if luatex or xetex
  \usepackage{unicode-math}
  \defaultfontfeatures{Scale=MatchLowercase}
  \defaultfontfeatures[\rmfamily]{Ligatures=TeX,Scale=1}
\fi
\usepackage{lmodern}
\ifPDFTeX\else  
    % xetex/luatex font selection
\fi
% Use upquote if available, for straight quotes in verbatim environments
\IfFileExists{upquote.sty}{\usepackage{upquote}}{}
\IfFileExists{microtype.sty}{% use microtype if available
  \usepackage[]{microtype}
  \UseMicrotypeSet[protrusion]{basicmath} % disable protrusion for tt fonts
}{}
\makeatletter
\@ifundefined{KOMAClassName}{% if non-KOMA class
  \IfFileExists{parskip.sty}{%
    \usepackage{parskip}
  }{% else
    \setlength{\parindent}{0pt}
    \setlength{\parskip}{6pt plus 2pt minus 1pt}}
}{% if KOMA class
  \KOMAoptions{parskip=half}}
\makeatother
\usepackage{xcolor}
\setlength{\emergencystretch}{3em} % prevent overfull lines
\setcounter{secnumdepth}{-\maxdimen} % remove section numbering
% Make \paragraph and \subparagraph free-standing
\makeatletter
\ifx\paragraph\undefined\else
  \let\oldparagraph\paragraph
  \renewcommand{\paragraph}{
    \@ifstar
      \xxxParagraphStar
      \xxxParagraphNoStar
  }
  \newcommand{\xxxParagraphStar}[1]{\oldparagraph*{#1}\mbox{}}
  \newcommand{\xxxParagraphNoStar}[1]{\oldparagraph{#1}\mbox{}}
\fi
\ifx\subparagraph\undefined\else
  \let\oldsubparagraph\subparagraph
  \renewcommand{\subparagraph}{
    \@ifstar
      \xxxSubParagraphStar
      \xxxSubParagraphNoStar
  }
  \newcommand{\xxxSubParagraphStar}[1]{\oldsubparagraph*{#1}\mbox{}}
  \newcommand{\xxxSubParagraphNoStar}[1]{\oldsubparagraph{#1}\mbox{}}
\fi
\makeatother

\usepackage{color}
\usepackage{fancyvrb}
\newcommand{\VerbBar}{|}
\newcommand{\VERB}{\Verb[commandchars=\\\{\}]}
\DefineVerbatimEnvironment{Highlighting}{Verbatim}{commandchars=\\\{\}}
% Add ',fontsize=\small' for more characters per line
\usepackage{framed}
\definecolor{shadecolor}{RGB}{241,243,245}
\newenvironment{Shaded}{\begin{snugshade}}{\end{snugshade}}
\newcommand{\AlertTok}[1]{\textcolor[rgb]{0.68,0.00,0.00}{#1}}
\newcommand{\AnnotationTok}[1]{\textcolor[rgb]{0.37,0.37,0.37}{#1}}
\newcommand{\AttributeTok}[1]{\textcolor[rgb]{0.40,0.45,0.13}{#1}}
\newcommand{\BaseNTok}[1]{\textcolor[rgb]{0.68,0.00,0.00}{#1}}
\newcommand{\BuiltInTok}[1]{\textcolor[rgb]{0.00,0.23,0.31}{#1}}
\newcommand{\CharTok}[1]{\textcolor[rgb]{0.13,0.47,0.30}{#1}}
\newcommand{\CommentTok}[1]{\textcolor[rgb]{0.37,0.37,0.37}{#1}}
\newcommand{\CommentVarTok}[1]{\textcolor[rgb]{0.37,0.37,0.37}{\textit{#1}}}
\newcommand{\ConstantTok}[1]{\textcolor[rgb]{0.56,0.35,0.01}{#1}}
\newcommand{\ControlFlowTok}[1]{\textcolor[rgb]{0.00,0.23,0.31}{\textbf{#1}}}
\newcommand{\DataTypeTok}[1]{\textcolor[rgb]{0.68,0.00,0.00}{#1}}
\newcommand{\DecValTok}[1]{\textcolor[rgb]{0.68,0.00,0.00}{#1}}
\newcommand{\DocumentationTok}[1]{\textcolor[rgb]{0.37,0.37,0.37}{\textit{#1}}}
\newcommand{\ErrorTok}[1]{\textcolor[rgb]{0.68,0.00,0.00}{#1}}
\newcommand{\ExtensionTok}[1]{\textcolor[rgb]{0.00,0.23,0.31}{#1}}
\newcommand{\FloatTok}[1]{\textcolor[rgb]{0.68,0.00,0.00}{#1}}
\newcommand{\FunctionTok}[1]{\textcolor[rgb]{0.28,0.35,0.67}{#1}}
\newcommand{\ImportTok}[1]{\textcolor[rgb]{0.00,0.46,0.62}{#1}}
\newcommand{\InformationTok}[1]{\textcolor[rgb]{0.37,0.37,0.37}{#1}}
\newcommand{\KeywordTok}[1]{\textcolor[rgb]{0.00,0.23,0.31}{\textbf{#1}}}
\newcommand{\NormalTok}[1]{\textcolor[rgb]{0.00,0.23,0.31}{#1}}
\newcommand{\OperatorTok}[1]{\textcolor[rgb]{0.37,0.37,0.37}{#1}}
\newcommand{\OtherTok}[1]{\textcolor[rgb]{0.00,0.23,0.31}{#1}}
\newcommand{\PreprocessorTok}[1]{\textcolor[rgb]{0.68,0.00,0.00}{#1}}
\newcommand{\RegionMarkerTok}[1]{\textcolor[rgb]{0.00,0.23,0.31}{#1}}
\newcommand{\SpecialCharTok}[1]{\textcolor[rgb]{0.37,0.37,0.37}{#1}}
\newcommand{\SpecialStringTok}[1]{\textcolor[rgb]{0.13,0.47,0.30}{#1}}
\newcommand{\StringTok}[1]{\textcolor[rgb]{0.13,0.47,0.30}{#1}}
\newcommand{\VariableTok}[1]{\textcolor[rgb]{0.07,0.07,0.07}{#1}}
\newcommand{\VerbatimStringTok}[1]{\textcolor[rgb]{0.13,0.47,0.30}{#1}}
\newcommand{\WarningTok}[1]{\textcolor[rgb]{0.37,0.37,0.37}{\textit{#1}}}

\providecommand{\tightlist}{%
  \setlength{\itemsep}{0pt}\setlength{\parskip}{0pt}}\usepackage{longtable,booktabs,array}
\usepackage{calc} % for calculating minipage widths
% Correct order of tables after \paragraph or \subparagraph
\usepackage{etoolbox}
\makeatletter
\patchcmd\longtable{\par}{\if@noskipsec\mbox{}\fi\par}{}{}
\makeatother
% Allow footnotes in longtable head/foot
\IfFileExists{footnotehyper.sty}{\usepackage{footnotehyper}}{\usepackage{footnote}}
\makesavenoteenv{longtable}
\usepackage{graphicx}
\makeatletter
\newsavebox\pandoc@box
\newcommand*\pandocbounded[1]{% scales image to fit in text height/width
  \sbox\pandoc@box{#1}%
  \Gscale@div\@tempa{\textheight}{\dimexpr\ht\pandoc@box+\dp\pandoc@box\relax}%
  \Gscale@div\@tempb{\linewidth}{\wd\pandoc@box}%
  \ifdim\@tempb\p@<\@tempa\p@\let\@tempa\@tempb\fi% select the smaller of both
  \ifdim\@tempa\p@<\p@\scalebox{\@tempa}{\usebox\pandoc@box}%
  \else\usebox{\pandoc@box}%
  \fi%
}
% Set default figure placement to htbp
\def\fps@figure{htbp}
\makeatother

\usepackage[margin=0.7in]{geometry}
\usepackage{fvextra}
\DefineVerbatimEnvironment{Highlighting}{Verbatim}{breaklines,commandchars=\\\{\}}
\KOMAoption{captions}{tableheading}
\makeatletter
\@ifpackageloaded{caption}{}{\usepackage{caption}}
\AtBeginDocument{%
\ifdefined\contentsname
  \renewcommand*\contentsname{Table of contents}
\else
  \newcommand\contentsname{Table of contents}
\fi
\ifdefined\listfigurename
  \renewcommand*\listfigurename{List of Figures}
\else
  \newcommand\listfigurename{List of Figures}
\fi
\ifdefined\listtablename
  \renewcommand*\listtablename{List of Tables}
\else
  \newcommand\listtablename{List of Tables}
\fi
\ifdefined\figurename
  \renewcommand*\figurename{Figure}
\else
  \newcommand\figurename{Figure}
\fi
\ifdefined\tablename
  \renewcommand*\tablename{Table}
\else
  \newcommand\tablename{Table}
\fi
}
\@ifpackageloaded{float}{}{\usepackage{float}}
\floatstyle{ruled}
\@ifundefined{c@chapter}{\newfloat{codelisting}{h}{lop}}{\newfloat{codelisting}{h}{lop}[chapter]}
\floatname{codelisting}{Listing}
\newcommand*\listoflistings{\listof{codelisting}{List of Listings}}
\makeatother
\makeatletter
\makeatother
\makeatletter
\@ifpackageloaded{caption}{}{\usepackage{caption}}
\@ifpackageloaded{subcaption}{}{\usepackage{subcaption}}
\makeatother

\usepackage{bookmark}

\IfFileExists{xurl.sty}{\usepackage{xurl}}{} % add URL line breaks if available
\urlstyle{same} % disable monospaced font for URLs
\hypersetup{
  pdftitle={Pset V},
  pdfauthor={Summer Negahdar \& Jenny Zhong},
  colorlinks=true,
  linkcolor={blue},
  filecolor={Maroon},
  citecolor={Blue},
  urlcolor={Blue},
  pdfcreator={LaTeX via pandoc}}


\title{Pset V}
\author{Summer Negahdar \& Jenny Zhong}
\date{}

\begin{document}
\maketitle

\RecustomVerbatimEnvironment{verbatim}{Verbatim}{
  showspaces = false,
  showtabs = false,
  breaksymbolleft={},
  breaklines
}


Partner 1: Summer Negahdar(samarneg) Partner 2: This submission is our
work alone and complies with the 30538 integrity policy.'' Add your
initials to indicate your agreement: **\_\_** \textbf{\textbf{\textbf{
``I have uploaded the names of anyone else other than my partner and I
worked with on the problem set here'' Late coins used this pset: }}}
Late coins left after submission: **\_\_**

\begin{Shaded}
\begin{Highlighting}[]
\ImportTok{from}\NormalTok{ bs4 }\ImportTok{import}\NormalTok{ BeautifulSoup}
\ImportTok{import}\NormalTok{ pandas }\ImportTok{as}\NormalTok{ pd}
\ImportTok{import}\NormalTok{ requests}
\end{Highlighting}
\end{Shaded}

\subsection{Develop Initial scraper and
crawler}\label{develop-initial-scraper-and-crawler}

\subsubsection{1.}\label{section}

\begin{Shaded}
\begin{Highlighting}[]
\CommentTok{\#doing the intial action to get the link parsed}
\NormalTok{url }\OperatorTok{=} \StringTok{\textquotesingle{}https://oig.hhs.gov/fraud/enforcement/\textquotesingle{}} \CommentTok{\# the link we would be scraping}
\NormalTok{requested }\OperatorTok{=}\NormalTok{ requests.get(url) }
\NormalTok{soup }\OperatorTok{=}\NormalTok{ BeautifulSoup(requested.content, }\StringTok{\textquotesingle{}html.parser\textquotesingle{}}\NormalTok{)}

\CommentTok{\# Find all actions based on the main \textless{}li\textgreater{} tag containing each card}
\NormalTok{actions }\OperatorTok{=}\NormalTok{ soup.find\_all(}\StringTok{\textquotesingle{}li\textquotesingle{}}\NormalTok{, class\_}\OperatorTok{=}\StringTok{\textquotesingle{}usa{-}card card{-}{-}list pep{-}card{-}{-}minimal mobile:grid{-}col{-}12\textquotesingle{}}\NormalTok{)}

\NormalTok{dataset }\OperatorTok{=}\NormalTok{ [] }\CommentTok{\#creating an ampty list to store craped data}

\ControlFlowTok{for}\NormalTok{ items }\KeywordTok{in}\NormalTok{ actions:}
\NormalTok{    title\_tag }\OperatorTok{=}\NormalTok{ items.find(}\StringTok{\textquotesingle{}h2\textquotesingle{}}\NormalTok{, class\_}\OperatorTok{=}\StringTok{\textquotesingle{}usa{-}card\_\_heading\textquotesingle{}}\NormalTok{) }\CommentTok{\#tag for the title of enforcement is h2}
    \ControlFlowTok{if}\NormalTok{ title\_tag:}
\NormalTok{        title }\OperatorTok{=}\NormalTok{ title\_tag.get\_text(strip}\OperatorTok{=}\VariableTok{True}\NormalTok{)  }
\NormalTok{        link }\OperatorTok{=}\NormalTok{ title\_tag.find(}\StringTok{\textquotesingle{}a\textquotesingle{}}\NormalTok{)[}\StringTok{\textquotesingle{}href\textquotesingle{}}\NormalTok{] }\ControlFlowTok{if}\NormalTok{ title\_tag.find(}\StringTok{\textquotesingle{}a\textquotesingle{}}\NormalTok{) }\ControlFlowTok{else} \VariableTok{None} \CommentTok{\#the tag for hyperlinks}
\NormalTok{        link }\OperatorTok{=} \SpecialStringTok{f"https://oig.hhs.gov}\SpecialCharTok{\{}\NormalTok{link}\SpecialCharTok{\}}\SpecialStringTok{"} \ControlFlowTok{if}\NormalTok{ link }\ControlFlowTok{else} \VariableTok{None}  \CommentTok{\# Complete relative link}

    \CommentTok{\#looking for dates}
\NormalTok{    date\_tag }\OperatorTok{=}\NormalTok{ items.find(}\KeywordTok{lambda}\NormalTok{ tag: tag.name }\OperatorTok{==} \StringTok{"span"} \KeywordTok{and} \StringTok{"text{-}base{-}dark"} \KeywordTok{in}\NormalTok{ tag.get(}\StringTok{"class"}\NormalTok{, []) }\KeywordTok{and} \StringTok{"padding{-}right{-}105"} \KeywordTok{in}\NormalTok{ tag.get(}\StringTok{"class"}\NormalTok{, []))}
\NormalTok{    date }\OperatorTok{=}\NormalTok{ date\_tag.get\_text(strip}\OperatorTok{=}\VariableTok{True}\NormalTok{) }\ControlFlowTok{if}\NormalTok{ date\_tag }\ControlFlowTok{else} \VariableTok{None}

    \CommentTok{\#now we will be looking for category}
\NormalTok{    category\_tag }\OperatorTok{=}\NormalTok{ items.find(}\KeywordTok{lambda}\NormalTok{ tag: tag.name }\OperatorTok{==} \StringTok{"ul"} \KeywordTok{and} \StringTok{"display{-}inline"} \KeywordTok{in}\NormalTok{ tag.get(}\StringTok{"class"}\NormalTok{, []) }\KeywordTok{and} \StringTok{"add{-}list{-}reset"} \KeywordTok{in}\NormalTok{ tag.get(}\StringTok{"class"}\NormalTok{, []))}
\NormalTok{    category }\OperatorTok{=} \VariableTok{None}
    \ControlFlowTok{if}\NormalTok{ category\_tag:}
\NormalTok{        li\_tag }\OperatorTok{=}\NormalTok{ category\_tag.find(}\KeywordTok{lambda}\NormalTok{ tag: tag.name }\OperatorTok{==} \StringTok{"li"} \KeywordTok{and} \StringTok{"display{-}inline{-}block"} \KeywordTok{in}\NormalTok{ tag.get(}\StringTok{"class"}\NormalTok{, []) }\KeywordTok{and} \StringTok{"usa{-}tag"} \KeywordTok{in}\NormalTok{ tag.get(}\StringTok{"class"}\NormalTok{, []))}
\NormalTok{        category }\OperatorTok{=}\NormalTok{ li\_tag.get\_text(strip}\OperatorTok{=}\VariableTok{True}\NormalTok{) }\ControlFlowTok{if}\NormalTok{ li\_tag }\ControlFlowTok{else} \VariableTok{None}

    \CommentTok{\# Append data to dataset}
\NormalTok{    dataset.append(\{}
        \StringTok{\textquotesingle{}Title\textquotesingle{}}\NormalTok{: title, }
        \StringTok{\textquotesingle{}Date\textquotesingle{}}\NormalTok{: date, }
        \StringTok{\textquotesingle{}Category\textquotesingle{}}\NormalTok{: category, }
        \StringTok{\textquotesingle{}Link\textquotesingle{}}\NormalTok{: link }
\NormalTok{    \})}

\NormalTok{final\_df }\OperatorTok{=}\NormalTok{ pd.DataFrame(dataset)}
\BuiltInTok{print}\NormalTok{(final\_df.head(}\DecValTok{5}\NormalTok{))}
\end{Highlighting}
\end{Shaded}

\begin{verbatim}
                                               Title              Date  \
0  Pharmacist and Brother Convicted of $15M Medic...  November 8, 2024   
1  Boise Nurse Practitioner Sentenced To 48 Month...  November 7, 2024   
2  Former Traveling Nurse Pleads Guilty To Tamper...  November 7, 2024   
3  Former Arlington Resident Sentenced To Prison ...  November 7, 2024   
4  Paroled Felon Sentenced To Six Years For Fraud...  November 7, 2024   

                     Category  \
0  Criminal and Civil Actions   
1  Criminal and Civil Actions   
2  Criminal and Civil Actions   
3  Criminal and Civil Actions   
4  Criminal and Civil Actions   

                                                Link  
0  https://oig.hhs.gov/fraud/enforcement/pharmaci...  
1  https://oig.hhs.gov/fraud/enforcement/boise-nu...  
2  https://oig.hhs.gov/fraud/enforcement/former-t...  
3  https://oig.hhs.gov/fraud/enforcement/former-a...  
4  https://oig.hhs.gov/fraud/enforcement/paroled-...  
\end{verbatim}

\subsubsection{2.}\label{section-1}

\begin{Shaded}
\begin{Highlighting}[]
\CommentTok{\# Initialize an empty list to store agency names}
\NormalTok{agencies }\OperatorTok{=}\NormalTok{ []}

\CommentTok{\# Loop through each link in final\_df}
\ControlFlowTok{for}\NormalTok{ index, row }\KeywordTok{in}\NormalTok{ final\_df.iterrows():}
\NormalTok{    link }\OperatorTok{=}\NormalTok{ row[}\StringTok{\textquotesingle{}Link\textquotesingle{}}\NormalTok{]}
    \ControlFlowTok{if}\NormalTok{ link:  }\CommentTok{\# Only proceed if the link is valid}
        \ControlFlowTok{try}\NormalTok{:}
\NormalTok{            response }\OperatorTok{=}\NormalTok{ requests.get(link)  }\CommentTok{\# Request the page using the link}
\NormalTok{            soup }\OperatorTok{=}\NormalTok{ BeautifulSoup(response.text, }\StringTok{\textquotesingle{}html.parser\textquotesingle{}}\NormalTok{)  }\CommentTok{\# Parse the content of the page}

            \CommentTok{\# Find all \textless{}ul\textgreater{} elements with the specified class containing the agency details}
\NormalTok{            uls }\OperatorTok{=}\NormalTok{ soup.find\_all(}\StringTok{"ul"}\NormalTok{, class\_}\OperatorTok{=}\StringTok{"usa{-}list usa{-}list{-}{-}unstyled margin{-}y{-}2"}\NormalTok{)}
            
            \CommentTok{\# Initialize a placeholder for the agency name}
\NormalTok{            agency\_name }\OperatorTok{=} \StringTok{\textquotesingle{}N/A\textquotesingle{}}
            
            \CommentTok{\# Iterate over each \textless{}ul\textgreater{} element}
            \ControlFlowTok{for}\NormalTok{ ul }\KeywordTok{in}\NormalTok{ uls:}
                \CommentTok{\# Find all \textless{}span\textgreater{} elements within each \textless{}ul\textgreater{} that match the class}
\NormalTok{                spans }\OperatorTok{=}\NormalTok{ ul.find\_all(}\StringTok{"span"}\NormalTok{, class\_}\OperatorTok{=}\StringTok{"padding{-}right{-}2 text{-}base"}\NormalTok{)}
                
                \CommentTok{\# Ensure there are enough \textless{}span\textgreater{} tags to avoid IndexError}
                \ControlFlowTok{if} \BuiltInTok{len}\NormalTok{(spans) }\OperatorTok{\textgreater{}} \DecValTok{1}\NormalTok{:}
\NormalTok{                    agency }\OperatorTok{=}\NormalTok{ spans[}\DecValTok{1}\NormalTok{]  }\CommentTok{\# Select the second \textless{}span\textgreater{}, which contains "Agency:"}
                    
                    \CommentTok{\# Use next\_sibling to access the text following the \textless{}span\textgreater{}}
\NormalTok{                    agency\_name }\OperatorTok{=}\NormalTok{ agency.next\_sibling.strip() }\ControlFlowTok{if}\NormalTok{ agency.next\_sibling }\ControlFlowTok{else} \StringTok{\textquotesingle{}N/A\textquotesingle{}}
                    
                    \CommentTok{\# Stop after finding the first matching \textless{}ul\textgreater{} and \textless{}span\textgreater{} structure}
                    \ControlFlowTok{break}
            
            \CommentTok{\# Append the extracted agency name to the agencies list}
\NormalTok{            agencies.append(agency\_name)}

        \ControlFlowTok{except}\NormalTok{ requests.exceptions.RequestException }\ImportTok{as}\NormalTok{ e:}
            \BuiltInTok{print}\NormalTok{(}\SpecialStringTok{f"Error fetching }\SpecialCharTok{\{}\NormalTok{link}\SpecialCharTok{\}}\SpecialStringTok{: }\SpecialCharTok{\{}\NormalTok{e}\SpecialCharTok{\}}\SpecialStringTok{"}\NormalTok{)}
\NormalTok{            agencies.append(}\StringTok{\textquotesingle{}N/A\textquotesingle{}}\NormalTok{)}

\CommentTok{\# Add agency names to the DataFrame and print its head}
\NormalTok{final\_df[}\StringTok{\textquotesingle{}Agency\textquotesingle{}}\NormalTok{] }\OperatorTok{=}\NormalTok{ agencies  }\CommentTok{\# Create a new column in our original df called Agency}
\BuiltInTok{print}\NormalTok{(final\_df.head(}\DecValTok{5}\NormalTok{))}
\end{Highlighting}
\end{Shaded}

\begin{verbatim}
                                               Title              Date  \
0  Pharmacist and Brother Convicted of $15M Medic...  November 8, 2024   
1  Boise Nurse Practitioner Sentenced To 48 Month...  November 7, 2024   
2  Former Traveling Nurse Pleads Guilty To Tamper...  November 7, 2024   
3  Former Arlington Resident Sentenced To Prison ...  November 7, 2024   
4  Paroled Felon Sentenced To Six Years For Fraud...  November 7, 2024   

                     Category  \
0  Criminal and Civil Actions   
1  Criminal and Civil Actions   
2  Criminal and Civil Actions   
3  Criminal and Civil Actions   
4  Criminal and Civil Actions   

                                                Link  \
0  https://oig.hhs.gov/fraud/enforcement/pharmaci...   
1  https://oig.hhs.gov/fraud/enforcement/boise-nu...   
2  https://oig.hhs.gov/fraud/enforcement/former-t...   
3  https://oig.hhs.gov/fraud/enforcement/former-a...   
4  https://oig.hhs.gov/fraud/enforcement/paroled-...   

                                              Agency  
0                         U.S. Department of Justice  
1  November 7, 2024; U.S. Attorney's Office, Dist...  
2  U.S. Attorney's Office, District of Massachusetts  
3  U.S. Attorney's Office, Eastern District of Vi...  
4  U.S. Attorney's Office, Middle District of Flo...  
\end{verbatim}

\begin{Shaded}
\begin{Highlighting}[]
\CommentTok{\#\#for jenny}
\CommentTok{\#\# since you need the date column to be a date, I will convert it for you}

\NormalTok{final\_df[}\StringTok{\textquotesingle{}Date\textquotesingle{}}\NormalTok{] }\OperatorTok{=}\NormalTok{ pd.to\_datetime(final\_df[}\StringTok{\textquotesingle{}Date\textquotesingle{}}\NormalTok{], errors}\OperatorTok{=}\StringTok{\textquotesingle{}coerce\textquotesingle{}}\NormalTok{)}

\CommentTok{\# Check the data type to confirm}
\BuiltInTok{print}\NormalTok{(final\_df.dtypes)}
\BuiltInTok{print}\NormalTok{(final\_df.head())}
\end{Highlighting}
\end{Shaded}

\begin{verbatim}
Title               object
Date        datetime64[ns]
Category            object
Link                object
Agency              object
dtype: object
                                               Title       Date  \
0  Pharmacist and Brother Convicted of $15M Medic... 2024-11-08   
1  Boise Nurse Practitioner Sentenced To 48 Month... 2024-11-07   
2  Former Traveling Nurse Pleads Guilty To Tamper... 2024-11-07   
3  Former Arlington Resident Sentenced To Prison ... 2024-11-07   
4  Paroled Felon Sentenced To Six Years For Fraud... 2024-11-07   

                     Category  \
0  Criminal and Civil Actions   
1  Criminal and Civil Actions   
2  Criminal and Civil Actions   
3  Criminal and Civil Actions   
4  Criminal and Civil Actions   

                                                Link  \
0  https://oig.hhs.gov/fraud/enforcement/pharmaci...   
1  https://oig.hhs.gov/fraud/enforcement/boise-nu...   
2  https://oig.hhs.gov/fraud/enforcement/former-t...   
3  https://oig.hhs.gov/fraud/enforcement/former-a...   
4  https://oig.hhs.gov/fraud/enforcement/paroled-...   

                                              Agency  
0                         U.S. Department of Justice  
1  November 7, 2024; U.S. Attorney's Office, Dist...  
2  U.S. Attorney's Office, District of Massachusetts  
3  U.S. Attorney's Office, Eastern District of Vi...  
4  U.S. Attorney's Office, Middle District of Flo...  
\end{verbatim}

\subsection{Making the scraper
dynamic}\label{making-the-scraper-dynamic}

\subsubsection{1.}\label{section-2}

the pseudo code for writing the function will be like: 1. going thorugh
every row in the df Summer has created.

\begin{enumerate}
\def\labelenumi{\arabic{enumi}.}
\setcounter{enumi}{1}
\item
  extracting the dates from date column
\item
  there will be two types of date

  I. after 2013 \textgreater\textgreater{} continue with the rest of
  function

  \begin{enumerate}
  \def\labelenumii{\Roman{enumii}.}
  \setcounter{enumii}{1}
  \tightlist
  \item
    before 2013 \textgreater\textgreater{} show me an error sign that
    says this is before our desired timeline
  \end{enumerate}
\item
  save all the extracted dates on a csv file called
  ``enfrocment\_actions\_month\_year.csv''
\item
  do not push it to git
\item
  print the head
\end{enumerate}

\begin{enumerate}
\def\labelenumi{\alph{enumi}.}
\tightlist
\item
\item
\item
\end{enumerate}

\subsection{Plot data based on scraped data (using
Altair)}\label{plot-data-based-on-scraped-data-using-altair}

\subsubsection{1.}\label{section-3}

\subsubsection{2.}\label{section-4}

\subsection{Create maps of enforcement
activity}\label{create-maps-of-enforcement-activity}

\subsubsection{1.}\label{section-5}

\subsubsection{2.}\label{section-6}

\subsection{Extra Credit: Calculate the enforcement actions on a
per-capita
basis}\label{extra-credit-calculate-the-enforcement-actions-on-a-per-capita-basis}

\subsubsection{1.}\label{section-7}

\subsubsection{2.}\label{section-8}

\subsubsection{3.}\label{section-9}




\end{document}
